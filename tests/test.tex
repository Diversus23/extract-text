\documentclass[12pt,a4paper]{article}

% Пакеты для поддержки русского языка
\usepackage[utf8]{inputenc}
\usepackage[russian]{babel}
\usepackage[T2A]{fontenc}

% Дополнительные пакеты
\usepackage{amsmath}
\usepackage{amsfonts}
\usepackage{amssymb}
\usepackage{graphicx}
\usepackage{hyperref}
\usepackage{geometry}
\usepackage{fancyhdr}
\usepackage{listings}
\usepackage{xcolor}
\usepackage{tabularx}
\usepackage{caption}
\usepackage{subcaption}

% Настройка геометрии страницы
\geometry{left=2.5cm,right=1.5cm,top=2cm,bottom=2cm}

% Настройка колонтитулов
\pagestyle{fancy}
\fancyhf{}
\fancyhead[L]{Система извлечения текста}
\fancyhead[R]{API v1.7}
\fancyfoot[C]{\thepage}

% Настройка для отображения кода
\lstset{
    language=Python,
    basicstyle=\ttfamily\small,
    keywordstyle=\color{blue},
    commentstyle=\color{green!50!black},
    stringstyle=\color{red},
    backgroundcolor=\color{gray!10},
    frame=single,
    breaklines=true,
    captionpos=b,
    numbers=left,
    numberstyle=\tiny\color{gray},
    showstringspaces=false
}

% Заголовок документа
\title{\textbf{Система извлечения текста из файлов различных форматов} \\
       \large Техническая документация API v1.7}
\author{ООО <<СОФТОНИТ>>}
\date{\today}

\begin{document}

\maketitle

\begin{abstract}
Данный документ описывает систему извлечения текста из файлов различных форматов, включая документы, изображения, таблицы и исходный код. Система предоставляет RESTful API для интеграции с внешними приложениями и поддерживает более 120 различных форматов файлов.
\end{abstract}

\tableofcontents
\newpage

\section{Введение}

Система извлечения текста представляет собой современное решение для обработки документов различных форматов. Основные возможности системы включают:

\begin{itemize}
    \item Извлечение текста из документов (PDF, DOCX, DOC, RTF)
    \item OCR распознавание текста на изображениях
    \item Обработка таблиц и электронных таблиц
    \item Поддержка исходного кода более 40 языков программирования
    \item Обработка электронных книг и писем
\end{itemize}

\section{Математические основы}

\subsection{Алгоритм OCR}

Точность распознавания текста определяется по формуле:

\begin{equation}
\text{Accuracy} = \frac{\text{Correctly recognized characters}}{\text{Total characters}} \times 100\%
\end{equation}

Где уверенность распознавания вычисляется как:

\begin{equation}
\text{Confidence} = \frac{1}{n} \sum_{i=1}^{n} c_i
\end{equation}

где $c_i$ -- уверенность распознавания $i$-го символа, $n$ -- общее количество символов.

\subsection{Обработка исходного кода}

Для анализа сложности кода используется метрика цикломатической сложности:

\begin{equation}
M = E - N + 2P
\end{equation}

где:
\begin{align}
E &= \text{количество рёбер в графе потока управления} \\
N &= \text{количество узлов в графе} \\
P &= \text{количество связанных компонентов}
\end{align}

\section{Архитектура системы}

\begin{figure}[htbp]
    \centering
    \begin{minipage}{0.8\textwidth}
        \begin{verbatim}
┌─────────────────┐    ┌─────────────────┐    ┌─────────────────┐
│   Client App    │────│   FastAPI       │────│   Extractors    │
│                 │    │   Server        │    │   Module        │
└─────────────────┘    └─────────────────┘    └─────────────────┘
                                │
                                │
                       ┌─────────────────┐
                       │   File Storage  │
                       │   & Processing  │
                       └─────────────────┘
        \end{verbatim}
    \end{minipage}
    \caption{Схема архитектуры системы}
    \label{fig:architecture}
\end{figure}

\section{Поддерживаемые форматы}

\begin{table}[htbp]
\centering
\caption{Основные категории поддерживаемых форматов}
\begin{tabularx}{\textwidth}{|l|X|c|}
\hline
\textbf{Категория} & \textbf{Описание} & \textbf{Количество} \\
\hline
Документы & PDF, DOCX, DOC, RTF, ODT & 5 \\
Изображения & JPG, PNG, TIFF, BMP, GIF & 7 \\
Таблицы & XLS, XLSX, CSV, ODS & 4 \\
Исходный код & Python, JavaScript, Java, C++, 1C и др. & 40+ \\
Структурированные & JSON, XML, YAML & 3 \\
\hline
\textbf{Всего} & & \textbf{120+} \\
\hline
\end{tabularx}
\label{tab:formats}
\end{table}

\section{Примеры кода}

\subsection{Использование API}

\begin{lstlisting}[caption=Пример запроса к API на Python]
import requests

# Загрузка файла
with open('document.pdf', 'rb') as file:
    response = requests.post(
        'http://localhost:7555/v1/extract-text/',
        files={'file': file}
    )

if response.status_code == 200:
    result = response.json()
    print(f"Извлеченный текст: {result['text']}")
    print(f"Количество слов: {result['word_count']}")
else:
    print(f"Ошибка: {response.status_code}")
\end{lstlisting}

\subsection{Обработка 1С кода}

\begin{lstlisting}[language=C, caption=Пример кода 1С:Предприятие]
// Функция обработки текста в 1С
Функция ОбработатьТекст(ВходнойТекст) Экспорт
    
    Если НЕ ЗначениеЗаполнено(ВходнойТекст) Тогда
        ВызватьИсключение "Текст не заполнен";
    КонецЕсли;
    
    Результат = Новый Структура;
    Результат.Вставить("ОбработанныйТекст", СокрЛП(ВходнойТекст));
    Результат.Вставить("КоличествоСимволов", СтрДлина(ВходнойТекст));
    
    Возврат Результат;
    
КонецФункции
\end{lstlisting}

\section{Производительность}

Система обеспечивает следующие характеристики производительности:

\begin{itemize}
    \item Максимальный размер файла: 20 МБ
    \item Время обработки документа: $< 30$ секунд
    \item Одновременных подключений: до 100
    \item Поддерживаемые языки OCR: русский, английский
\end{itemize}

\section{Формулы вычислений}

\subsection{Статистика текста}

Количество слов в тексте:
\[
W = |\{w_1, w_2, \ldots, w_n\}|
\]

Плотность ключевых слов:
\[
\text{Density} = \frac{\text{Keyword count}}{\text{Total words}} \times 100\%
\]

\subsection{Качество распознавания}

Индекс качества OCR:
\[
Q = \alpha \cdot \text{Accuracy} + \beta \cdot \text{Confidence} + \gamma \cdot \text{Speed}
\]

где $\alpha + \beta + \gamma = 1$ и $\alpha, \beta, \gamma \geq 0$.

\section{Заключение}

Система извлечения текста представляет собой комплексное решение для обработки документов различных форматов. Благодаря модульной архитектуре и поддержке множества форматов, система может эффективно использоваться в различных сценариях обработки данных.

\bibliographystyle{plain}
\begin{thebibliography}{9}

\bibitem{fastapi}
Sebastian Ramirez.
\textit{FastAPI framework, high performance, easy to learn, fast to code, ready for production}.
\url{https://fastapi.tiangolo.com/}, 2018-2024.

\bibitem{tesseract}
Ray Smith.
\textit{An Overview of the Tesseract OCR Engine}.
Proceedings of the Ninth International Conference on Document Analysis and Recognition, 2007.

\bibitem{pypdf}
Mathieu Fenniak.
\textit{PyPDF2: A utility to read and write PDFs with Python}.
\url{https://github.com/py-pdf/PyPDF2}, 2006-2024.

\end{thebibliography}

\end{document} 